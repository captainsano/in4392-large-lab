\documentclass[a4paper, sigconf, twocolumn]{acmart}

\usepackage{lipsum}

\begin{document}

% {\raggedright{
% Parallel and Distributed Systems Group\\
% Faculty of Electrical Engineering, Mathematics and Computer Science\\
% Delft University of Technology
% }}

\begin{center}
  \textbf{\LARGE{
    Imagely: An elastic cloud environment for online image processing
  }}\\
  \vspace{0.25cm}
  \emph{Authors}: SB Ramalingam Santhanakrishnan (4740270), \\ K Kleeberger () and B Jain ()\\
  ICT Innovation, EEMCS, TU Delft\\
  \emph{Emails}: \{S.B.RamalingamSanthanakrishnan, B.Jain, K.Kleeberger\}@student.tudelft.nl\\
  \vspace{0.2cm}
  \emph{Course Instructors}: Alexandru Iosup and Dick Epema\\
  PDS Group, EEMCS, TU Delft\\
  \emph{Emails}: \{A.Iosup, D.H.J.Epema\}@tudelft.nl\\
  \vspace{0.2cm}
  \emph{Lab Assistant}: Bogdan Ghit\\
  PDS Group, EEMCS, TU Delft\\
  \emph{Email}: B.I.Ghit@tudelft.nl\\
\end{center}

\vspace{0.2cm}

\textbf{
  \emph{Abstract}---In this report we introduce \emph{Imagely}, a cloud service for image processing.
  It runs on top of Amazon Web Services compute resources to provide elastic scaling capabilities and
  we show that it can achieve a speedup of upto x.xx at peak load, compared to baseline performance.
}

\section{Introduction}

\section{Application}

The application which we build is a web-based image manipulator which
supports basic operations on the given image(s) such as crop, border, 
frame, trim, chop, draw, annotate, resize, scale, magnify, etc. 
A sequence of operations on the same image can be specified in a single
request. We use the ImageMagick\cite{imagemagick} application for the
aforementioned operations and wrap it with a thin NodeJS\cite{nodejs} web server.

\section{System Design}

\subsection{Resource Management Architecture}

\subsection{System Policies}

\subsection{Additional System Features [OPTIONAL]}

\section{Experimental Results}

\subsection{Experimental Setup}

\subsection{Experiments}

\begin{enumerate}
  \item{\emph{Charged-time}: }
  \item{\emph{Charged-cost}: }
  \item{\emph{Service metrics of the experiment}: }
  \item{\emph{Usage metrics of the experiment (OPTIONAL)}: }
\end{enumerate}

\section{Conclusion}

\bibliographystyle{unsrt}
\bibliography{large-lab}

\section*{Appendix A: Time Sheets}

[TODO: Restructure this section onto a table]

\textbf{Project:}
\begin{enumerate}
  \item{\emph{Total time}: }
  \item{\emph{Think time}: }
  \item{\emph{Dev time}: }
  \item{\emph{XP time}: }
  \item{\emph{Analysis time}: }
  \item{\emph{Write time}: }
  \item{\emph{Wasted time}: }
\end{enumerate}

\textbf{Per Experiment:}
\begin{enumerate}
  \item{\emph{Total time}: }
  \item{\emph{Dev time}: }
  \item{\emph{Setup time}: }
\end{enumerate}

\end{document}